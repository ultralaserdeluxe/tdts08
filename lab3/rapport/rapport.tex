\documentclass[titlepage, a4paper]{article}
\usepackage[english]{babel}
\usepackage[utf8]{inputenc}
\usepackage{graphicx}
\usepackage{color}
\usepackage{mathtools}
\usepackage{float}
\usepackage[parfill]{parskip}
\usepackage[margin=10pt,font=small,labelfont=bf,labelsep=endash]{caption}
\usepackage{epstopdf}
\usepackage{listings}
\usepackage[table]{xcolor}
\epstopdfsetup{suffix=}
\DeclareGraphicsExtensions{.ps}
\DeclareGraphicsRule{.ps}{pdf}{.pdf}{`ps2pdf -dEPSCrop -dNOSAFER #1 \noexpand\OutputFile}


\definecolor{green}{rgb}{56,90,115}

\lstset{literate=%
    {å}{{\r{a}}}1
    {ä}{{\"a}}1
    {ö}{{\"o}}1
    {Å}{{\r{A}}}1
    {Ä}{{\"A}}1
    {Ö}{{\"O}}1
}

\newcommand{\todo}[1] {\textbf{\textcolor{red}{#1}}}

\usepackage{fancyhdr}
\fancyhead[L]{}
\pagestyle{fancy}
\rhead{Alexander Yngve \\ Pål Kastman}
\chead{TDTS08}
\thispagestyle{empty}

\begin{document}

{\ }\vspace{45mm}

\begin{center}
  \Huge \textbf{TDTS08: Lab Report}
\end{center}
\begin{center}
  \Large Lab 3: Superscalar Processors
\end{center}

\vspace{250pt}

\begin{center}
  \begin{tabular}{|*{3}{p{40mm}|}}
    \hline
    \textbf{Name} & \textbf{PIN} & \textbf{Email} \\ \hline
           {Alexander Yngve} & {930320-6651} & {aleyn573@student.liu.se} \\ \hline
           {Pål Kastman} & {851212-7575} & {palka285@student.liu.se} \\ \hline
  \end{tabular}
\end{center}
\newpage

\tableofcontents
\thispagestyle{empty}
\newpage

\section{Introduction}\label{sec:intro}
The purpose of this lab is to learn how Supersclar Processors work, and to try and modify an processor architecture to make it simpler, but it should still perform within 5\% of the inital designs performance.

\section{Method}
We started out by investigating every part of the design individually, to see how they affected the performance of the design.

We then choose to simplify the parts that didn't affect the performance. We determined what parts we couldn't simplify due to that the performance would go further than 5\% from the initial performance.

Now we looked at the parts of the design that we could modify, and at their traces.

\section{Result}
In order to establish a baseline performance the simulator was run with the default arguments and a trace was created, with the command shown in listing \ref{sim:simcom}.

\begin{lstlisting}[caption=Simulator command., label=sim:simcom, breaklines=true]
sim-outorder -config superscalar.cfg -ptrace trace.trc 100000:+30 go.ss 3 8
\end{lstlisting}

The config file ''superscalar.cfg'' is supplied with the lab and contains all default settings. The most interesting of these are shown in table \ref{tab:default}.

\begin{table}[H]
\centering
\caption{Default settings.}

\begin{tabular}{|l|r|}
  \hline
  \textbf{Setting} & \textbf{Value} \\ \hline
  res:ialu & 4 \\ \hline
  res:imult & 4 \\ \hline
  res:fpalu & 2 \\ \hline
  res:fpmult & 2 \\ \hline
  ruu:size & 32 \\ \hline
  commit:width & 8 \\ \hline
  issue:width & 8 \\ \hline
  decode:width & 16 \\ \hline
  fetch:speed & 1 \\ \hline
\end{tabular}

\label{tab:default}
\end{table}

Running the simulator with the default settings gave us a simulation time of \textbf{50868222} cycles, with the 5\% requirement from section \ref{sec:intro} this results in a maximum of \textbf{53411633} simulation cycles.

\subsection{Integer components}


\begin{table}[H]
\centering
\caption{Integer settings.}

\begin{tabular}{|l|r|r|}
  \hline
  \textbf{Setting} & \textbf{Value} & \textbf{Simulation cycles}\\ \hline
  res:ialu & 2 & 55360220 \\ \hline
  res:ialu & 1 & 73570279 \\ \hline
  res:imult & 2 & 50868222 \\ \hline
  res:imult & 1 & 50868609 \\ \hline
\end{tabular}

\label{tab:integer}
\end{table}

\subsection{Floating Point components}
\todo{insert graphs of how the parameters changed} \\
In figure \ref{} we can see that by changing the floating point alu \& multiplier, the system didn't perform any worse, thus these parts can be simplified as much as possible.

\begin{table}[H]
\centering
\caption{Floating point settings.}

\begin{tabular}{|l|r|r|}
  \hline
  \textbf{Setting} & \textbf{Value} & \textbf{Simulation cycles}\\ \hline
  res:fpalu & 1 & 50868222 \\ \hline
  res:fpmult & 1 & 50868222 \\ \hline
\end{tabular}

\label{tab:floatingpoint}
\end{table}

\subsection{Control components}
The speed of the system was already at the lowest possible, which means by changing this value we get a more complex design, thus we decided not to.

\begin{table}[H]
\centering
\caption{Control components settings.}

\begin{tabular}{|l|r|r|}
  \hline
  \textbf{Setting} & \textbf{Value} & \textbf{Simulation cycles}\\ \hline
  ruu:size & 16 & 53807041 \\ \hline
  ruu:size & 8 & 60781481 \\ \hline
  ruu:size & 4 & 74098784 \\ \hline
  ruu:size & 2 & 107565825 \\ \hline
  commit:width & 4 & 51076751 \\ \hline
  commit:width & 2 & 53040135 \\ \hline
  commit:width & 1 & 67227355 \\ \hline
  issue:width & 4 & 51187526 \\ \hline
  issue:width & 2 & 59378979 \\ \hline
  issue:width & 1 & 85571238 \\ \hline
  decode:width & 8 & 51165249 \\ \hline
  decode:width & 4 & 53609877 \\ \hline
  decode:width & 2 & 61683477 \\ \hline
  fetch:speed & 4 & 49804290 \\ \hline
  fetch:speed & 2 & 50868222 \\ \hline
\end{tabular}

\label{tab:floatingpoint}
\end{table}

\section{Discussion}
\todo{explain why go.ss doesn't need those} \\

\end{document}

%\begin{figure}[H]
%	\centering
%	\scalebox{0.342}{\includegraphics{}}
%	\caption{}
%	\label{fig:}
%\end{figure}
