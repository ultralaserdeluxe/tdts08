\section{Lecture 9: MIMD Architectures}
A set of general purpose processors connected together \\
In contrast to a SIMD computer, a MIMD computer can execute different programs on different processors. \\

-- Works asynchronously, and don't have to synchronize with each other. \\
-- At any time, different processors may be executing different instructions on different pieces of data. \\
-- They can be built from commodity (off-the-shelf) microprocessors with relatively little effort. \\
-- They are also highly scalable, provided that an appropiate memory organization is used. \\
-- Most current parallel computer are built based on the MIMD architecture. \\

\subsubsection{SIMD vs. MIMD}
\subsubsection{MIMD Processor Classification}
\subsubsection{MIMD with Shared Memory}
\subsubsection{MIMD with Distributed Memory}
\subsubsection{Shared-Address-Space Platforms}
\subsubsection{Multi-Computer Systems}
\subsubsection{MIM Design Issues}
\subsection{Symmetric multiprocessors (SMP)}
\subsubsection{SMP Advantages}
\subsubsection{SMP based on Shared Bus}
\subsubsection{Multi-Port Memory SMP}
\subsubsection{Operation System Issues}
\subsubsection{IBM S/390}
\subsection{NUMA Architecture}
\subsubsection{Memory Access Approaches}
\subsection{Clusters}
\subsubsection{Losely Coupled MIMD - Clusters}
\subsubsection{Clusters benefits}
\subsubsection{Clusters Configurations}
\subsubsection{IBM Blue Gene Supercomputer}
\subsubsection{Parallelizing Computation}
\subsubsection{Google Applications}
