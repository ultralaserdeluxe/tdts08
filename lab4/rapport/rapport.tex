\documentclass[titlepage, a4paper]{article}
\usepackage[english]{babel}
\usepackage[utf8]{inputenc}
\usepackage{graphicx}
\usepackage{color}
\usepackage{mathtools}
\usepackage{float}
\usepackage[parfill]{parskip}
\usepackage[margin=10pt,font=small,labelfont=bf,labelsep=endash]{caption}
\usepackage{epstopdf}
\usepackage{listings}
\usepackage[table]{xcolor}
\usepackage{tabularx}
\epstopdfsetup{suffix=}
\DeclareGraphicsExtensions{.ps}
\DeclareGraphicsRule{.ps}{pdf}{.pdf}{`ps2pdf -dEPSCrop -dNOSAFER #1 \noexpand\OutputFile}


\definecolor{green}{rgb}{56,90,115}

\lstset{literate=%
    {å}{{\r{a}}}1
    {ä}{{\"a}}1
    {ö}{{\"o}}1
    {Å}{{\r{A}}}1
    {Ä}{{\"A}}1
    {Ö}{{\"O}}1
}

\newcommand{\todo}[1] {\textbf{\textcolor{red}{#1}}}

\usepackage{fancyhdr}
\fancyhead[L]{}
\pagestyle{fancy}
\rhead{Alexander Yngve \\ Pål Kastman}
\chead{TDTS08}
\thispagestyle{empty}

\begin{document}

{\ }\vspace{45mm}

\begin{center}
  \Huge \textbf{TDTS08: Lab Report}
\end{center}
\begin{center}
  \Large Lab 4: VLIW Processors
\end{center}

\vspace{250pt}

\begin{center}
  \begin{tabular}{|*{3}{p{40mm}|}}
    \hline
    \textbf{Name} & \textbf{PIN} & \textbf{Email} \\ \hline
           {Alexander Yngve} & {930320-6651} & {aleyn573@student.liu.se} \\ \hline
           {Pål Kastman} & {851212-7575} & {palka285@student.liu.se} \\ \hline
  \end{tabular}
\end{center}
\newpage

\tableofcontents
\thispagestyle{empty}
\newpage

\section{Introduction}\label{sec:intro}
The purpose of this lab was to convert normal sequential code to VLIW instructions, so that we would get a greater performance. Basically we do what the VLIW compiler does during compilation time.
\section{Method}
The approach for this lab was the following:
\begin{enumerate}
\item Choose a basic block, and disassemble the block. 
\item Find dependencies between the instruction in the block.
\item We pack the instruction into VLIWs.  
\end{enumerate}

\subsection{Basic Block}
\subsection{Dependencies}
\subsection{VLIW}
\section{Result}
\section{Discussion}
\end{document}
